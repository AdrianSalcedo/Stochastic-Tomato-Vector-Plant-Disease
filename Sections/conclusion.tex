\paragraph{Findings}
1. We established by algebraic manipulations basic reproductive number
$\mathcal{R}^s_0$ for \autoref{sys::StochasticSystem}. We observe that
$\mathcal{R}^s_0$ is a threshold parameter as the deterministic case.
\paragraph{Judgment}
2. With this threshold parameter, we can establish preventive measures to reduce
the spread of the disease in the crop. With this, we obtain a greater production.
In the definition of $\mathcal{R}^s_0$ it is observed that the most significant
measures are replanting and fumigation. If we take into account increasing said
prevention measures against the disease we will obtain a more productive crop.
\paragraph{Limitations}
One of the complications found is the fact of getting the most out of replanting
for latent and infected plants. 
\paragraph{Perspectives}
4. suggestions for improvements (perhaps in relation to the limitations)
To improve this, I would like to investigate some technique to obtain the
stochastic basic reproductive number in terms of each replanting rate for plants.

5. recommendations for future work (either for the author, and/or the community)
For a future work, I would like to apply the theory of stochastic
control to \autoref{sys::StochasticSystem} be able to give control
measures over the disease, based on which strategy is most beneficial
and why to carry it out.

