%!TEX root = ../main.tex
	In this section we will study when the disease can be extinguished, for 
this we will give the necessary conditions so that this phenomenon can occur 
through two different cases. The first case will be when due to the intensity 
of the noise.The theorem presented below shows that under conditions on the
parameters we can make the disease tend to become extinct.
%\change{Change definition to taalkabout FDE point}
\begin{definition}\label{def::ExponentialStability}
The free-disease fixed point of \autoref{sys::StochasticSystem} is said to be almost surely exponentially stable if
\begin{equation}\label{eqn::ExponentialStability}
		\limsup_{t \to \infty}
		\frac{1}{t}
		\ln(L_p + I_p) < 0 \text{ and }
		\limsup_{t \to \infty}
		\frac{1}{t}\ln(I_v)< 0\qquad\mbox{a.s.}
\end{equation}
\end{definition}

\begin{theorem}\label{thm::NoiseExtinction}
	Let	any initial condition $(S_p(0), L_p(0), I_p(0), S_v(0), I_v(0)) ^ \top \in \Gamma$. If
	\begin{align*}
		\frac{\beta_p ^ 2}{2\sigma_L ^ 2} + 
		\frac{r_2^2}{2\sigma_I^2} + 
		2 \beta_p - r_1 < 0,
		\qquad
		\frac{\beta_v^2}{2\sigma_v^2} + 
		\beta_v - \gamma + \theta \mu < 0,
	\end{align*}
	 then the disease will exponentially extinguish with probability one. That 
	 is, 
	\begin{equation*}
		\limsup_{t \to \infty}
		\frac{1}{t}
		\ln(L_p + I_p) < 0 \text{ and }
		\limsup_{t \to \infty}
		\frac{1}{t}\ln(I_v)< 0\qquad\mbox{a.s.}
	\end{equation*}
\end{theorem}
\begin{proof}
	The main idea is apply the It\^{o} formula to a conveniently function and 
	deduce conditions. Let 
	$
		V(S_p, L_p, I_p) = \ln(L_p + I_p)
	$, then the It\^{o} formula gives
	\begin{align*}
		d \ln(L_p+I_p) 
			=&
				\left(
					\frac{1}{L_p + I_p}
				\right)
				\left(
					\frac{\beta_p}{N_v^\infty} 
					S_p I_v - (b + r_1) L_p
					-\frac{1}{2}
					\sigma_L^2 \frac{L_p^2}{(L_p+I_p)^2}
				\right)dt
			\\
			&-
				\sigma_L \frac{L_p}{L_p+I_p}dB_p(t)\\
			&\leq 
				\left(
					\frac{1}{L_p+I_p}
				\right)
				\left(
					\beta_p S_p - (b + r_1) -
					\frac{1}{2}
					\sigma_L^2
					\frac{L_p^2}{(L_p+I_p)^2}
				\right)dt
			\\
			&-
				\sigma_L \frac{L_p}{L_p + I_p} dB_p(t).
			\\
	\end{align*}
	Let $x:=\frac{L_p}{L_p + I_p}$, then
		\begin{align*}
		d \ln(L_p + I_p) 
			&\leq 
				\left(
					\beta_p 
					\frac{S_p}{L_p + I_p} - 
					(b + r_1) - 
					\frac{1}{2}
					\sigma_L ^ 2 x^2
				\right)dt - \sigma_L x dB_p(t)
			\\
			&\leq
				\left(\beta_p
					\frac{N_p}{L_p + I_p} - (b+r_1) - 
					\frac{1}{2} 
					\sigma_L^2 x^2
				\right) dt - 
				\sigma_L xdB_p(t)
			\\
			&\leq
				\left(
					\beta_p x + 2\beta_p - 
					(b + r_1) - 
					\frac{1}{2}
					\sigma_L ^ 2 x^2
				\right) dt - 
				\sigma_L xdB_p(t)
			\\
			&=
				\left(
					-\frac{1}{2}
					\sigma_L ^ 2 x ^ 2 + 
					\beta_p x + 2 \beta_p - 
					(b + r_1)
				\right) dt -\sigma_L x 
				dB_p(t).			
	\end{align*}
	Hence,
	\begin{align*}
		\ln(L_p+I_p)
			&\leq
				-\frac{\sigma_L ^ 2}{2}
				\int_{0} ^ {t}
					\left[
						\left(
							x - 
							\frac{\beta_p}{\sigma_L ^ 2}
						\right) ^ 2 + 
						\frac{\beta_p ^ 2}{2 \sigma_L ^ 2} + 
						2 \beta_p - (b + r_1)
					\right] du
				\\
			&-
				\int_{0} ^ {t}
					\sigma_L x dB_p(u) + 
					\ln(L_p(0) + I_p(0)),
	\end{align*}
	which implies,
	\begin{equation}
	\label{eqn::ItoForBound}
		\begin{aligned}
			\frac{1}{t}\ln(L_p+I_p) 
				&\leq
					-\frac{\sigma_L^2}{2t}
					\int_{0}^{t}
					\left(
						x - 
						\frac{\beta_p}{\sigma_L^2}
					\right) ^ 2 du + 
					\frac{\beta_p^2}{2\sigma_L^2} - 
					(b + r_1) + 2\beta_p
				\\
				&-
					\frac{1}{t}
					\int_{0}^{t}
					\sigma_L x dB_p(u) + 
					\frac{1}{t} \ln(S_p(0)+L_p(0)+I_p(0)),
		\end{aligned}
	\end{equation}
	let 
	$$
	M_t :=
		\frac{1}{t}
		\int_{0}^{t}
			\sigma_L x dB_p(t) + 
			\frac{1}{t} \ln(L_p(0)+I_p(0)) .
	$$ 
	Since the integral in the term $M_t$ is a martingale, the strong law of
	large numbers for martingales \cite[see][Theorem 3.4, pp.12]{Mao2008}, implies that
	\begin{equation*}
		\lim
		\limits_{t \to \infty} M_t = 0\,\,
		\mbox{a.s.}
	\end{equation*}
	Thus, from relation \autoref{eqn::ItoForBound} we obtain
	\begin{align}
		\label{eqn::Bound1}
		\limsup_{t\infty \to \infty}
		\frac{1}{t}
		\ln(L_p + I_p) < 
			\frac{\beta_p^2}{2\sigma_L^2} +
			2\beta_p - (b + r_1)
	\end{align}
	A similar argument also shows that
	\begin{align}\label{eqn::Bound2}
		\limsup_{t\infty \to \infty}
		\frac{1}{t}
		\ln(L_p + I_p) < 
		\frac{r_2 ^ 2}{2 \sigma_I ^ 2} + b
	\end{align}
%
	Through by \autoref{eqn::Bound1} and \autoref{eqn::Bound2}, we obtain
%	
	\begin{align*}
		\limsup_{t\infty \to \infty}
		\frac{1}{t}
		\ln(L_p + I_p) 
			< 
			\frac{\beta_p^2}{2\sigma_L^2} + 
			\frac{r_2^2}{2 \sigma_I ^ 2} +	
			2\beta_p - r_1
	\end{align*}
	and for infected vector we obtain
	\begin{align*}
		\limsup_{t\infty \to \infty}
		\frac{1}{t}
		\ln(I_v) 
			< 
			\frac{\beta_v^2}{2\sigma_v^2} 
			+ 
			\beta_v - \gamma + \theta \mu
	\end{align*}
	\qed
\end{proof}
%
\begin{remark}
\autoref{thm::NoiseExtinction} shows that, under certain conditions on the
parameters can cause disease exponentially towards zero whenever the noise
intensity is large enough.
\end{remark}

To define the stochastic reproductive number we will use techniques similar to
those used in\cite{Agarwal2019}, in which, by means of algebraic procedures,
this parameter can be defined. 

To define the basic stochastic reproductive number, let's define 
$ r = \max \{r_1, r_2 \}$. 
Then the basic reproductive number of the system \autoref{sys::StochasticSystem}
is defined as

\begin{equation}\label{eqn::StochasticBRN}
	\mathcal{R}_0 ^ s =
		\frac{\beta_p \beta_v}{\gamma r}.
\end{equation}
%
The following theorem proves that under $\mathcal{R}^s_0<1$, the infected plants
and vectors tends to zero.
%\change{Rewrite according to the above notation}
\begin{theorem}
	\label{thm::Rs0Extinction}
	Let $(S_p(t),L_p(t),I_p(t),I_v(t))^\top$ 
	be the solution of SDE \autoref{sys::StochasticSystem} with initial values 
	$
		(S_p(0),L_p(0),I_p(0),I_v(0))^\top \in \Gamma
	$. 
	If $0\leq \mathcal{R}^s_0<1$, 
	then the following conditions holds
	\begin{align*}
		\lim
		\limits_{t\to \infty}
		\frac{1}{t}
		\mathbb{E}
		\int_{0}^{t}
			\left[
				r
				[\mathcal{R}^s_0 - 1]
				I_p - r S_p
				\left(
					1 - \frac{S^0_p}{S_p}
				\right) ^ 2 - 
				r L_p - 
				\frac{\beta_p\beta_v}{\gamma} 
				I_v I_p
			\right] dr 
			\leq \frac{1}{2} \sigma_p^ 2 N_p,\, a.s.,
	\end{align*}
	namely, the infected individual tends to zero exponentially a.s, i.e the 
	disease will die out with probability one.
\end{theorem}
%
\begin{proof}
	The proof consist verify the hypotheses of Khasminskii Theorem 
	\change{Fix citation}
	\info{Write Khasminskii Theorem in appendix}
	\cite{Agarwal2019} for 
	the Lyapunov function
	%
	\begin{align*}
		V(S_p, L_p, I_p, S_v, I_v) 
			=& 
				\left(
					S_p - N_p - N_p 
					\ln 
					\frac{S_p}{N_p}
				\right) + 
				L_p + I_p + 
				\frac{\beta_p N_p}{\gamma N^\infty_v} I_v
			\\
			&+
				\left(
					S_v - N_v - N_v 
					\ln
					\frac{S_v}{N_v}
				\right).
	\end{align*}
	%
	Let $f$, $g$ respectively be the drift and diffusion of SDE 
	\autoref{sys::StochasticSystem}. Applying the diffusion operator
	$\mathcal{L}$ we have
	%
	\begin{equation}\label{eqn::LyapunovFunction}
		\begin{aligned}
			V_x f 
				&\leq
					\left(
						1 - 
						\frac{N_p}{S_p}
					\right)
					\left(
						-
						\frac{\beta_p}{N^\infty_v} S_p I_v + 
						r N_p - r S_p
					\right) + 
					\frac{\beta_p}{N^\infty_v} S_p I_v - 
					(b + r) L_p
				\\
				&+
					b L_p - r I_p + 
					\left(
						1 - 
						\frac{N_v}{S_v}
					\right)
					\left( 
						-\frac{\beta_v}{N_p} S_v I_p - 
						\gamma S_v + 
						(1 - \theta) \mu 
					\right)
				\\
				&+
					\frac{\beta_p N_p}{\gamma N ^ \infty_v}
					\left(
						\frac{\beta_v S_v}{N_p}
						I_p - \gamma I_v + 
						\theta \mu
					\right)  \ .
		\end{aligned}
	\end{equation}
	Expanded the term $\left(
				1 - \frac{N_p}{S_p}
			\right)
			\left(
				-\frac{\beta_p}{N^\infty_v} S_p I_v +
				r N_p - r S_p
			\right)$ 
			of \autoref{eqn::LyapunovFunction} and factoring the term $S_p$,
			we obtain
	\begin{equation}
		\label{theorem2term1}
		\begin{aligned}
			\left(
				1 - \frac{N_p}{S_p}
			\right)
			&
			\left(
				-\frac{\beta_p}{N^\infty_v} S_p I_v +
				r N_p - r S_p
			\right) 	
			\\
			&=
				\left(
						1 - \frac{N_p}{S_p}
					\right)
					\left(- r S_p 
						\left(
							1 - \frac{N_p}{S_p}
						\right) - 
						\frac{\beta_p}{N^\infty_v} S_p I_v
					\right)
			\\
			&=
				- r S_p 
				\left(
					1 - 
					\frac{N_p}{S_p}
				\right) ^ 2 - 
				\frac{\beta_p}{N ^ \infty_v} S_p I_v + 
				\frac{\beta_p}{N ^ \infty_v} N_p I_v .
		\end{aligned}
	\end{equation}
	%\improvement{rewrite}
	For the term $\left(
				1 - 
				\frac{N_v}{S_v}
			\right)
			\left(
				-\frac{\beta_v}{N_p} S_v I_p - 
				\gamma S_v + 
				(1 - \theta) \mu 
			\right) $,
			since $\theta\in [0,1]$ and $(1-\theta)\mu\leq \gamma N_v$
			we bound by the following
	\begin{equation}\label{theorem2term2}
		\begin{aligned}
			\left(
				1 - 
				\frac{N_v}{S_v}
			\right)
			&
			\left(
				-\frac{\beta_v}{N_p} S_v I_p - 
				\gamma S_v + 
				(1 - \theta) \mu 
			\right) 
			\\
			&
			\leq
				\left(
					1 - \frac{N_v}{S_v}
				\right)
				\left(- 
					\frac{\beta_v}{N_p} S_v I_p -
					\gamma S_v +\gamma N_v
				\right)
			\\
			&\leq
				\left(
					1 - \frac{N_v}{S_v}
				\right)
				\left(-
					\gamma S_v
					\left(
						1 - 
						\frac{N_v}{S_v}
					\right) -
					\frac{\beta_v}{N_p} S_v I_p
				\right)
			\\
			&\leq
				-\gamma S_v 
				\left(
					1- 
					\frac{N_v}{S_v}
				\right) ^ 2 - 
				\frac{\beta_v}{N_p} S_v I_p + 
				\frac{\beta_v}{N_p} N_v I_p \ .
		\end{aligned}
	\end{equation}
	Same way from above calculation, and since 
	$\theta \mu\leq \theta\gamma N_v$, we obtain
	\begin{equation}
		\label{theorem2term3}
		\begin{aligned}
			\frac{\beta_p N_p}{\gamma N^\infty_v}
			\left(
				\frac{\beta_v S_v}{N_p} I_p- 
				\gamma I_v + \theta \mu
			\right)
			&\leq
				\frac{\beta_p N_p}{\gamma N^\infty_v}
				\left(
					\frac{\beta_v S_v}{N_p} I_p -
					\gamma I_v + 
					\theta \gamma N_v
				\right) 
			\\
			&\leq
				\frac{\beta_p \beta_v S_v I_p}{\gamma N_v} - 
				\frac{\beta_p N_p}{ N^\infty_v} I_v + 
				\beta_p \theta N_p.
		\end{aligned}
	\end{equation}
	\begin{align*}
		V_x f 
			& \leq
				- r S_p 
				\left(
					1 - 
					\frac{N_p}{S_p}
				\right) ^ 2 + 
				\left[
					\frac{\beta_p}{N^\infty_v} N_p - 
					\frac{\beta_p N_p}{ N^\infty_v}
				\right] I_v - 
				r (L_p + I_p)
			\\
				& -
				\gamma S_v
				\left(
					1 - 
					\frac{N_v}{S_v}
				\right) ^ 2 - 
				\frac{\beta_v}{N_p} S_v I_p + 
				\frac{\beta_v}{N_p} N_v I_p
			\\
				& +
				\frac{\beta_p\beta_v S_v I_p}{\gamma N_v} + 
				\beta_p \theta N_p \ .
	\end{align*}
	%
	Moreover, since $S_v+I_v\leq N_v$, we can obtain the following relation
	%
	\begin{align*}
		V_x f 
			\leq &
				- r S_p 
				\left(
					1 - 
					\frac{N_p}{S_p}
				\right) ^ 2 - 
				r (L_p + I_p)
			\\
			&-
				\gamma S_v
				\left(1 - 
					\frac{N_v}{S_v}
				\right) ^ 2 + 
				\frac{\beta_v}{N_p} I_v I_p
			\\
			&+
				\frac{\beta_p\beta_v I_p}{\gamma} - 
				\frac{\beta_p\beta_v I_v I_p}{\gamma N_v} + 
				\beta_p \theta N_p \ .
	\end{align*}
%
	Expressing the right hand side of above equation in term of the basic 
	reproductive number, $\mathcal{R}^s_0$ we get
	\begin{align*}
		V_x f 
			&\leq
				-rS_p 
				\left(
					1 - 
					\frac{S_p^0}{S_p}
				\right) ^ 2 -
				\gamma S_v
				\left(
					1 - 
					\frac{N_v}{S_v}
				\right) ^ 2 - 
				r L_p - r
				\left[
					1 - \mathcal{R}^s_0
				\right] I_p
			\\
			&-
				\left[
					\frac{\beta_p\beta_v}{\gamma N^\infty_v} - 
					\frac{\beta_v}{N_p} 
				\right] I_v I_p - 
				\frac{\beta_v}{N_p} S_v I_p + 
				\beta_p \theta N_p \ .
	\end{align*}
	%
	Moreover,
	%
	\begin{align*}
		\frac{1}{2}
		\trace(g ^ TV_{xx} g) 
		&=
		\frac{1}{2} 
		\frac{
			\left(
				 \sigma_L L_p + \sigma_I I_p
			\right) ^ 2
		}{N_p} 
		+ 
		\frac{1}{2} \sigma_v ^ 2 N_v,
	\end{align*}
	%
	define $\sigma_p = \max\{\sigma_L,\sigma_I\}$, then
	
		\begin{align*}
		\frac{1}{2}
		\trace(g ^ TV_{xx} g) 
		&\leq
		\frac{1}{2} 
		\sigma_p ^ 2 N_p +  
		\frac{1}{2} \sigma_v ^ 2 N_v \ .
	\end{align*}
	
	The stochastic terms are not necessary, because they do a martingale process
	and by the Strong law of large numbers, Theorem 3.4 
	\cite[see][pp. 12]{Mao2008},when we use integral and expectation they
	vanishing. Incorporation all terms calculate above, we obtain
	\begin{align*}
	 	\mathcal{L}V(X) 
			&\leq
				-r S_p
				\left(
					1 -\frac{S_p^0}{S_p}
				\right) ^ 2 - 
				\gamma S_v
				\left(
					1 - 
					\frac{N_v}{S_v}
				\right) ^ 2 - 
				r L_p - r 
				\left[
					1 - \mathcal{R}^s_0
				\right] I_p
			\\
			&-
				\left[
					\frac{\beta_p \beta_v}{\gamma N ^ \infty_v} - 
					\frac{\beta_v}{N_p}
				\right] I_v I_p - 
				\frac{\beta_v}{N_p} S_v I_p + 
				\beta_p \theta N_p + 
				\frac{1}{2} 
				\sigma_p ^ 2 N_p + 
				\frac{1}{2}
				\sigma_v^2 N_v\ .
			\\
	\end{align*}
	%
	Define 
	$
		\sigma_{p,v}:= 
			\beta_p \theta N_p + 
			\frac{1}{2} 
			\sigma_p ^ 2 N_p +  
			\frac{1}{2} \sigma_v ^ 2 N_v
	$, then
	\begin{align*}
		\mathcal{L}V(X) 
			&\leq
				-r S_p
				\left(
					1 - 
					\frac{S_p^0}{S_p}
				\right) ^ 2 - 
				\gamma S_v
				\left( 1 -
					\frac{N_v}{S_v}
				\right) ^ 2 - r L_p - r 
				\left[
					1 - \mathcal{R}^s_0
				\right] I_p
			\\
			&-
				\left[
					\frac{\beta_p\beta_v}{\gamma N^\infty_v} - 
					\frac{\beta_v}{N_p}
				\right] I_v I_p - 
				\frac{\beta_v}{N_p} S_v I_p + 
				\sigma_{p,v} \ .
	\end{align*}
	%
	Since $V(x)\geq 0$, using the integral form of It\^{o}'s formula and 
	taking expectation yields
	\begin{align*}
		0 
		&\leq
			\mathbb{E}V(t) - 
			\mathbb{E} V(0)
			\leq
			\mathbb{E}
			\int_{0} ^ {t}
				\mathcal{L} V(X(s)) ds
			\\
		&\leq
			-\mathbb{E}
			\int_{0} ^ {t}
			\left[
				rS_p 
				\left(
					1 - \frac{S_p^0}{S_p}
				\right)^2 + 
				\gamma S_v
				\left(1 - 
					\frac{N_v}{S_v}
				\right) ^ 2 + 
				r L_p + 
				r 
				\left[
					1 - \mathcal{R} ^ s _ 0
				\right] I_p 
			\right.
		\\
		&+
			\left.
				\left[
					\frac{\beta_p\beta_v}{\gamma N^\infty_v} + 
					\frac{\beta_v}{N_p}
				\right] I_v I_p + 
				\frac{\beta_v}{N_p} S_v I_p - 
				\sigma_{p,v}
			\right] ds \ .
	\end{align*}
%
	Therefore,
%	
	\begin{align*}
		\frac{1}{t}
		\mathbb{E}
			&\int_{0}^{t}
				\left[
					rS_p 
					\left(
						1 - \frac{S_p^0}{S_p}
					\right) ^ 2 +
					\gamma S_v
					\left(
						1 - \frac{N_v}{S_v}
					\right) ^ 2 + 
					r L_p + 
					r 
					\left[
						1 - \mathcal{R} ^ s_ 0
					\right]I_p 
				\right.
			\\
			&+
				\left.
					\left[
						\frac{\beta_p \beta_v}{\gamma N ^ \infty_v} + 
						\frac{\beta_v}{N_p}
					\right] I_v I_p + 
					\frac{\beta_v}{N_p}S_vI_p
				\right] ds 
				\leq
				\sigma_{p,v} \ .
	\end{align*}
	This implies that,
	\begin{align*}
		\lim\limits_{t\to \infty}
		\frac{1}{t} \mathbb{E}
			&
			\int_{0}^{t}
			\left[
				r S_p
				\left(
					1 - 
					\frac{S_p^0}{S_p}
				\right) ^ 2 + 
				\gamma S_v
				\left(
					1 - \frac{N_v}{S_v}
				\right) ^ 2 + 
				r L_p + 
				r 
				\left[
					1 - \mathcal{R} ^ s _ 0
				\right] I_p 
			\right.
		\\
		&+
		\left.
			\left[
				\frac{\beta_p \beta_v}{\gamma N ^ \infty_v} + 
				\frac{\beta_v}{N_p}
			\right] I_v I_p + 
			\frac{\beta_v}{N_p}S_v I_p
		\right] ds 
		\leq 
		\sigma_{p,v}.
	\end{align*}
	%
	Taking 
	$\theta$, $\sigma_p$, and $\sigma_v$ such that $0<\sigma_{p,v}< 1$, we 
	have
	\begin{multline*}
		\lim%
		\limits_{t\to \infty}
		\frac{1}{t}
		\log
		\mathbb{E}
		\int_{0} ^ {t}
			%
			\left[
				r S_p
				\left(
					1-
					\frac{S_p ^ 0}{S_p}
				\right) ^ 2 +
			\right.
			%
			\gamma S_v
			\left(
				1 - 
				\frac{N_v }{ S_v}
			\right) ^ 2 + 
			%& 
			r L_p + r
			\left[
				1 - \mathcal{R}^s_0
			\right] I_p 
		\\
			 +
			%	
			\left(
				\frac{\beta_p\beta_v}{\gamma N^\infty_v} + 
				\frac{\beta_v}{N_p}
			\right) I_v I_p + 
		 %\\	
			%& 
			\left.
				\frac{\beta_v}{N_p} S_v I_p
			\right] 
			ds \leq 
			\log 
				\sigma_{p,v} <0 \ .
	\end{multline*}
	Therefore,
	\begin{multline*}
		\lim
		\limits_{t\to\infty}
		\mathbb{E}
			\int_{0} ^ {t}
		 		\left[ 
		 			r S_p
		 			\left(
		 				1 - 
		 				\frac{S_p^0}{S_p}
		 			\right)^ 2 +
		 		\right.
		 		\gamma S_v
		 		\left(
		 			1 - 
		 			\frac{N_v}{S_v}
		 		\right) ^ 2 + 	
				r L_p 
		 			+ 
		 			r
		 			\left[
		 				1 - \mathcal{R}^s_0
		 			\right] I_p 
				+
				\\
				\left.
					\left[
						\frac{\beta_p \beta_v}{\gamma N ^ \infty_v} + 
						\frac{\beta_v}{N_p}
					\right] I_v I_p + 
					\frac{\beta_v}{N_p} S_v I_p
				\right] 
				ds 
				\leq
				\lim
				\limits_{t\to \infty}
					e^{\sigma_{p,v} t } = 0 \ .
	\end{multline*}
%		\improvement{rewrite}
Letting $t \to \infty$ we obtain
	\begin{align*}
		\lim\limits_{t \to\infty}
			(S_p, L_p, I_p, S_v, I_v )^{\top}_{t}
			=
			(N_p, 0, 0, N_v, 0)
	\end{align*}
		exponentially a.s. \qed
\end{proof}
\begin{remark}
	Theorem \ref{thm::Rs0Extinction} shows that, if the basic stochastic
	reproductive number $\mathcal{R}^s_0$ is less than one, we have the 
	solutions $X(t)=(S_p (t), L_p (t), (t ) I_p (t), S_v (t), I_v (t))^{\top}$ 
	tend to the equilibrium point $(N_p, 0,0, N_v^{\infty}, 0)^{\top}$, 
	when $t\rightarrow \infty$.
\end{remark}